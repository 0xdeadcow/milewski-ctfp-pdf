\begin{quote}
This is part 21 of Categories for Programmers. Previously:
\href{https://bartoszmilewski.com/2016/11/21/monads-programmers-definition/}{Monads:
Programmer's Definition}. See the
\href{https://bartoszmilewski.com/2014/10/28/category-theory-for-programmers-the-preface/}{Table
of Contents}.
\end{quote}

Now that we know what the monad is for --- it lets us compose
embellished functions --- the really interesting question is why
embellished functions are so important in functional programming. We've
already seen one example, the \code{Writer} monad, where embellishment
let us create and accumulate a log across multiple function calls. A
problem that would otherwise be solved using impure functions (e.g., by
accessing and modifying some global state) was solved with pure
functions.

\section{The Problem}\label{the-problem}

Here is a short list of similar problems, copied from
\href{https://core.ac.uk/download/pdf/21173011.pdf}{Eugenio Moggi's
seminal paper}, all of which are traditionally solved by abandoning the
purity of functions.

\begin{itemize}
\tightlist
\item
  Partiality: Computations that may not terminate
\item
  Nondeterminism: Computations that may return many results
\item
  Side effects: Computations that access/modify state

  \begin{itemize}
  \tightlist
  \item
    Read-only state, or the environment
  \item
    Write-only state, or a log
  \item
    Read/write state
  \end{itemize}
\item
  Exceptions: Partial functions that may fail
\item
  Continuations: Ability to save state of the program and then restore
  it on demand
\item
  Interactive Input
\item
  Interactive Output
\end{itemize}

What really is mind blowing is that all these problems may be solved
using the same clever trick: turning to embellished functions. Of
course, the embellishment will be totally different in each case.

You have to realize that, at this stage, there is no requirement that
the embellishment be monadic. It's only when we insist on composition
--- being able to decompose a single embellished function into smaller
embellished functions --- that we need a monad. Again, since each of the
embellishments is different, monadic composition will be implemented
differently, but the overall pattern is the same. It's a very simple
pattern: composition that is associative and equipped with identity.

The next section is heavy on Haskell examples. Feel free to skim or even
skip it if you're eager to get back to category theory or if you're
already familiar with Haskell's implementation of monads.

\section{The Solution}\label{the-solution}

First, let's analyze the way we used the \code{Writer} monad. We
started with a pure function that performed a certain task --- given
arguments, it produced a certain output. We replaced this function with
another function that embellished the original output by pairing it with
a string. That was our solution to the logging problem.

We couldn't stop there because, in general, we don't want to deal with
monolithic solutions. We needed to be able to decompose one
log-producing function into smaller log-producing functions. It's the
composition of those smaller functions that led us to the concept of a
monad.

What's really amazing is that the same pattern of embellishing the
function return types works for a large variety of problems that
normally would require abandoning purity. Let's go through our list and
identify the embellishment that applies to each problem in turn.

\subsubsection{Partiality}\label{partiality}

We modify the return type of every function that may not terminate by
turning it into a ``lifted'' type --- a type that contains all values of
the original type plus the special ``bottom'' value \code{⊥}. For
instance, the \code{Bool} type, as a set, would contain two elements:
\code{True} and \code{False}. The lifted \code{Bool} contains
three elements. Functions that return the lifted \code{Bool} may
produce \code{True} or \code{False}, or execute forever.

The funny thing is that, in a lazy language like Haskell, a never-ending
function may actually return a value, and this value may be passed to
the next function. We call this special value the bottom. As long as
this value is not explicitly needed (for instance, to be pattern
matched, or produced as output), it may be passed around without
stalling the execution of the program. Because every Haskell function
may be potentially non-terminating, all types in Haskell are assumed to
be lifted. This is why we often talk about the category \textbf{Hask} of
Haskell (lifted) types and functions rather than the simpler
\textbf{Set}. It is not clear, though, that \textbf{Hask} is a real
category (see this
\href{http://math.andrej.com/2016/08/06/hask-is-not-a-category/}{Andrej
Bauer post}).

\subsubsection{Nondeterminism}\label{nondeterminism}

If a function can return many different results, it may as well return
them all at once. Semantically, a non-deterministic function is
equivalent to a function that returns a list of results. This makes a
lot of sense in a lazy garbage-collected language. For instance, if all
you need is one value, you can just take the head of the list, and the
tail will never be evaluated. If you need a random value, use a random
number generator to pick the n-th element of the list. Laziness even
allows you to return an infinite list of results.

In the list monad --- Haskell's implementation of nondeterministic
computations --- \code{join} is implemented as \code{concat}.
Remember that \code{join} is supposed to flatten a container of
containers --- \code{concat} concatenates a list of lists into a
single list. \code{return} creates a singleton list:

\begin{verbatim}
instance Monad [] where join = concat return x = [x]
\end{verbatim}

The bind operator for the list monad is given by the general formula:
\code{fmap} followed by \code{join} which, in this case gives:

\begin{verbatim}
as >>= k = concat (fmap k as)
\end{verbatim}

Here, the function \code{k}, which itself produces a list, is applied
to every element of the list \code{as}. The result is a list of lists,
which is flattened using \code{concat}.

From the programmer's point of view, working with a list is easier than,
for instance, calling a non-deterministic function in a loop, or
implementing a function that returns an iterator (although,
\href{http://ericniebler.com/2014/04/27/range-comprehensions/}{in modern
C++}, returning a lazy range would be almost equivalent to returning a
list in Haskell).

A good example of using non-determinism creatively is in game
programming. For instance, when a computer plays chess against a human,
it can't predict the opponent's next move. It can, however, generate a
list of all possible moves and analyze them one by one. Similarly, a
non-deterministic parser may generate a list of all possible parses for
a given expression.

Even though we may interpret functions returning lists as
non-deterministic, the applications of the list monad are much wider.
That's because stitching together computations that produce lists is a
perfect functional substitute for iterative constructs --- loops ---
that are used in imperative programming. A single loop can be often
rewritten using \code{fmap} that applies the body of the loop to each
element of the list. The \code{do} notation in the list monad can be
used to replace complex nested loops.

My favorite example is the program that generates Pythagorean triples
--- triples of positive integers that can form sides of right triangles.

\begin{verbatim}
triples = do z <- [1..] x <- [1..z] y <- [x..z] guard (x^2 + y^2 == z^2) return (x, y, z)
\end{verbatim}

The first line tells us that \code{z} gets an element from an infinite
list of positive numbers \code{{[}1..{]}}. Then \code{x} gets an
element from the (finite) list \code{{[}1..z{]}} of numbers between 1
and \code{z}. Finally \code{y} gets an element from the list of
numbers between \code{x} and \code{z}. We have three numbers
\code{1\ \textless{}=\ x\ \textless{}=\ y\ \textless{}=\ z} at our
disposal. The function \code{guard} takes a \code{Bool} expression
and returns a list of units:

\begin{verbatim}
guard :: Bool -> [()] guard True = [()] guard False = []
\end{verbatim}

This function (which is a member of a larger class called
\code{MonadPlus}) is used here to filter out non-Pythagorean triples.
Indeed, if you look at the implementation of bind (or the related
operator \code{\textgreater{}\textgreater{}}), you'll notice that,
when given an empty list, it produces an empty list. On the other hand,
when given a non-empty list (here, the singleton list containing unit
\code{{[}(){]}}), bind will call the continuation, here
\code{return\ (x,\ y,\ z)}, which produces a singleton list with a
verified Pythagorean triple. All those singleton lists will be
concatenated by the enclosing binds to produce the final (infinite)
result. Of course, the caller of \code{triples} will never be able to
consume the whole list, but that doesn't matter, because Haskell is
lazy.

The problem that normally would require a set of three nested loops has
been dramatically simplified with the help of the list monad and the
\code{do} notation. As if that weren't enough, Haskell let's you
simplify this code even further using list comprehension:

\begin{verbatim}
triples = [(x, y, z) | z <- [1..] , x <- [1..z] , y <- [x..z] , x^2 + y^2 == z^2]
\end{verbatim}

This is just further syntactic sugar for the list monad (strictly
speaking, \code{MonadPlus}).

You might see similar constructs in other functional or imperative
languages under the guise of generators and coroutines.

\subsubsection{Read-Only State}\label{read-only-state}

A function that has read-only access to some external state, or
environment, can be always replaced by a function that takes that
environment as an additional argument. A pure function
\code{(a,\ e)\ -\textgreater{}\ b} (where \code{e} is the type of
the environment) doesn't look, at first sight, like a Kleisli arrow. But
as soon as we curry it to
\code{a\ -\textgreater{}\ (e\ -\textgreater{}\ b)} we recognize the
embellishment as our old friend the reader functor:

\begin{verbatim}
newtype Reader e a = Reader (e -> a)
\end{verbatim}

You may interpret a function returning a \code{Reader} as producing a
mini-executable: an action that given an environment produces the
desired result. There is a helper function \code{runReader} to execute
such an action:

\begin{verbatim}
runReader :: Reader e a -> e -> a runReader (Reader f) e = f e
\end{verbatim}

It may produce different results for different values of the
environment.

Notice that both the function returning a \code{Reader}, and the
\code{Reader} action itself are pure.

To implement bind for the \code{Reader} monad, first notice that you
have to produce a function that takes the environment \code{e} and
produces a \code{b}:

\begin{verbatim}
ra >>= k = Reader (\e -> ...)
\end{verbatim}

Inside the lambda, we can execute the action \code{ra} to produce an
\code{a}:

\begin{verbatim}
ra >>= k = Reader (\e -> let a = runReader ra e in ...)
\end{verbatim}

We can then pass the \code{a} to the continuation \code{k} to get a
new action \code{rb}:

\begin{verbatim}
ra >>= k = Reader (\e -> let a = runReader ra e rb = k a in ...)
\end{verbatim}

Finally, we can run the action \code{rb} with the environment
\code{e}:

\begin{verbatim}
ra >>= k = Reader (\e -> let a = runReader ra e rb = k a in runReader rb e)
\end{verbatim}

To implement \code{return} we create an action that ignores the
environment and returns the unchanged value.

Putting it all together, after a few simplifications, we get the
following definition:

\begin{verbatim}
instance Monad (Reader e) where ra >>= k = Reader (\e -> runReader (k (runReader ra e)) e) return x = Reader (\e -> x)
\end{verbatim}

\subsubsection{Write-Only State}\label{write-only-state}

This is just our initial logging example. The embellishment is given by
the \code{Writer} functor:

\begin{verbatim}
newtype Writer w a = Writer (a, w)
\end{verbatim}

For completeness, there's also a trivial helper \code{runWriter} that
unpacks the data constructor:

\begin{verbatim}
runWriter :: Writer w a -> (a, w) runWriter (Writer (a, w)) = (a, w)
\end{verbatim}

As we've seen before, in order to make \code{Writer} composable,
\code{w} has to be a monoid. Here's the monad instance for
\code{Writer} written in terms of the bind operator:

\begin{verbatim}
instance (Monoid w) => Monad (Writer w) where (Writer (a, w)) >>= k = let (a&apos;, w&apos;) = runWriter (k a) in Writer (a&apos;, w `mappend` w&apos;) return a = Writer (a, mempty)
\end{verbatim}

\subsubsection{State}\label{state}

Functions that have read/write access to state combine the
embellishments of the \code{Reader} and the \code{Writer}. You may
think of them as pure functions that take the state as an extra argument
and produce a pair value/state as a result:
\code{(a,\ s)\ -\textgreater{}\ (b,\ s)}. After currying, we get them
into the form of Kleisli arrows
\code{a\ -\textgreater{}\ (s\ -\textgreater{}\ (b,\ s))}, with the
embellishment abstracted in the \code{State} functor:

\begin{verbatim}
newtype State s a = State (s -> (a, s))
\end{verbatim}

Again, we can look at a Kleisli arrow as returning an action, which can
be executed using the helper function:

\begin{verbatim}
runState :: State s a -> s -> (a, s) runState (State f) s = f s
\end{verbatim}

Different initial states may not only produce different results, but
also different final states.

The implementation of bind for the \code{State} monad is very similar
to that of the \code{Reader} monad, except that care has to be taken
to pass the correct state at each step:

\begin{verbatim}
sa >>= k = State (\s -> let (a, s&apos;) = runState sa s sb = k a in runState sb s&apos;)
\end{verbatim}

Here's the full instance:

\begin{verbatim}
instance Monad (State s) where sa >>= k = State (\s -> let (a, s&apos;) = runState sa s in runState (k a) s&apos;) return a = State (\s -> (a, s))
\end{verbatim}

There are also two helper Kleisli arrows that may be used to manipulate
the state. One of them retrieves the state for inspection:

\begin{verbatim}
get :: State s s get = State (\s -> (s, s))
\end{verbatim}

and the other replaces it with a completely new state:

\begin{verbatim}
put :: s -> State s () put s&apos; = State (\s -> ((), s&apos;))
\end{verbatim}

\subsubsection{Exceptions}\label{exceptions}

An imperative function that throws an exception is really a partial
function --- it's a function that's not defined for some values of its
arguments. The simplest implementation of exceptions in terms of pure
total functions uses the \code{Maybe} functor. A partial function is
extended to a total function that returns \code{Just\ a} whenever it
makes sense, and \code{Nothing} when it doesn't. If we want to also
return some information about the cause of the failure, we can use the
\code{Either} functor instead (with the first type fixed, for
instance, to \code{String}).

Here's the \code{Monad} instance for \code{Maybe}:

\begin{verbatim}
instance Monad Maybe where Nothing >>= k = Nothing Just a >>= k = k a return a = Just a
\end{verbatim}

Notice that monadic composition for \code{Maybe} correctly
short-circuits the computation (the continuation \code{k} is never
called) when an error is detected. That's the behavior we expect from
exceptions.

\subsubsection{Continuations}\label{continuations}

It's the ``Don't call us, we'll call you!'' situation you may experience
after a job interview. Instead of getting a direct answer, you are
supposed to provide a handler, a function to be called with the result.
This style of programming is especially useful when the result is not
known at the time of the call because, for instance, it's being
evaluated by another thread or delivered from a remote web site. A
Kleisli arrow in this case returns a function that accepts a handler,
which represents ``the rest of the computation'':

\begin{verbatim}
data Cont r a = Cont ((a -> r) -> r)
\end{verbatim}

The handler \code{a\ -\textgreater{}\ r}, when it's eventually called,
produces the result of type \code{r}, and this result is returned at
the end. A continuation is parameterized by the result type. (In
practice, this is often some kind of status indicator.)

There is also a helper function for executing the action returned by the
Kleisli arrow. It takes the handler and passes it to the continuation:

\begin{verbatim}
runCont :: Cont r a -> (a -> r) -> r runCont (Cont k) h = k h
\end{verbatim}

The composition of continuations is notoriously difficult, so its
handling through a monad and, in particular, the \code{do} notation,
is of extreme advantage.

Let's figure out the implementation of bind. First let's look at the
stripped down signature:

\begin{verbatim}
(>>=) :: ((a -> r) -> r) -> (a -> (b -> r) -> r) -> ((b -> r) -> r)
\end{verbatim}

Our goal is to create a function that takes the handler
\code{(b\ -\textgreater{}\ r)} and produces the result \code{r}. So
that's our starting point:

\begin{verbatim}
ka >>= kab = Cont (\hb -> ...)
\end{verbatim}

Inside the lambda, we want to call the function \code{ka} with the
appropriate handler that represents the rest of the computation. We'll
implement this handler as a lambda:

\begin{verbatim}
runCont ka (\a -> ...)
\end{verbatim}

In this case, the rest of the computation involves first calling
\code{kab} with \code{a}, and then passing \code{hb} to the
resulting action \code{kb}:

\begin{verbatim}
runCont ka (\a -> let kb = kab a in runCont kb hb)
\end{verbatim}

As you can see, continuations are composed inside out. The final handler
\code{hb} is called from the innermost layer of the computation.
Here's the full instance:

\begin{verbatim}
instance Monad (Cont r) where ka >>= kab = Cont (\hb -> runCont ka (\a -> runCont (kab a) hb)) return a = Cont (\ha -> ha a)
\end{verbatim}

\subsubsection{Interactive Input}\label{interactive-input}

This is the trickiest problem and a source of a lot of confusion.
Clearly, a function like \code{getChar}, if it were to return a
character typed at the keyboard, couldn't be pure. But what if it
returned the character inside a container? As long as there was no way
of extracting the character from this container, we could claim that the
function is pure. Every time you call \code{getChar} it would return
exactly the same container. Conceptually, this container would contain
the superposition of all possible characters.

If you're familiar with quantum mechanics, you should have no problem
understanding this analogy. It's just like the box with the
Schrödinger's cat inside --- except that there is no way to open or peek
inside the box. The box is defined using the special built-in
\code{IO} functor. In our example, \code{getChar} could be declared
as a Kleisli arrow:

\begin{verbatim}
getChar :: () -> IO Char
\end{verbatim}

(Actually, since a function from the unit type is equivalent to picking
a value of the return type, the declaration of \code{getChar} is
simplified to \code{getChar\ ::\ IO\ Char}.)

Being a functor, \code{IO} lets you manipulate its contents using
\code{fmap}. And, as a functor, it can store the contents of any type,
not just a character. The real utility of this approach comes to light
when you consider that, in Haskell, \code{IO} is a monad. It means
that you are able to compose Kleisli arrows that produce \code{IO}
objects.

You might think that Kleisli composition would allow you to peek at the
contents of the \code{IO} object (thus ``collapsing the wave
function,'' if we were to continue the quantum analogy). Indeed, you
could compose \code{getChar} with another Kleisli arrow that takes a
character and, say, converts it to an integer. The catch is that this
second Kleisli arrow could only return this integer as an
\code{(IO\ Int)}. Again, you'll end up with a superposition of all
possible integers. And so on. The Schrödinger's cat is never out of the
bag. Once you are inside the \code{IO} monad, there is no way out of
it. There is no equivalent of \code{runState} or \code{runReader}
for the \code{IO} monad. There is no \code{runIO}!

So what can you do with the result of a Kleisli arrow, the \code{IO}
object, other than compose it with another Kleisli arrow? Well, you can
return it from \code{main}. In Haskell, \code{main} has the
signature:

\begin{verbatim}
main :: IO ()
\end{verbatim}

and you are free to think of it as a Kleisli arrow:

\begin{verbatim}
main :: () -> IO ()
\end{verbatim}

From that perspective, a Haskell program is just one big Kleisli arrow
in the \code{IO} monad. You can compose it from smaller Kleisli arrows
using monadic composition. It's up to the runtime system to do something
with the resulting \code{IO} object (also called \code{IO} action).

Notice that the arrow itself is a pure function --- it's pure functions
all the way down. The dirty work is relegated to the system. When it
finally executes the \code{IO} action returned from \code{main}, it
does all kinds of nasty things like reading user input, modifying files,
printing obnoxious messages, formatting a disk, and so on. The Haskell
program never dirties its hands (well, except when it calls
\code{unsafePerformIO}, but that's a different story).

Of course, because Haskell is lazy, \code{main} returns almost
immediately, and the dirty work begins right away. It's during the
execution of the \code{IO} action that the results of pure
computations are requested and evaluated on demand. So, in reality, the
execution of a program is an interleaving of pure (Haskell) and dirty
(system) code.

There is an alternative interpretation of the \code{IO} monad that is
even more bizarre but makes perfect sense as a mathematical model. It
treats the whole Universe as an object in a program. Notice that,
conceptually, the imperative model treats the Universe as an external
global object, so procedures that perform I/O have side effects by
virtue of interacting with that object. They can both read and modify
the state of the Universe.

We already know how to deal with state in functional programming --- we
use the state monad. Unlike simple state, however, the state of the
Universe cannot be easily described using standard data structures. But
we don't have to, as long as we never directly interact with it. It's
enough that we assume that there exists a type \code{RealWorld} and,
by some miracle of cosmic engineering, the runtime is able to provide an
object of this type. An \code{IO} action is just a function:

\begin{verbatim}
type IO a = RealWorld -> (a, RealWorld)
\end{verbatim}

Or, in terms of the \code{State} monad:

\begin{verbatim}
type IO = State RealWorld
\end{verbatim}

However, \code{\textgreater{}=\textgreater{}} and \code{return} for
the \code{IO} monad have to be built into the language.

\subsubsection{Interactive Output}\label{interactive-output}

The same \code{IO} monad is used to encapsulate interactive output.
\code{RealWorld} is supposed to contain all output devices. You might
wonder why we can't just call output functions from Haskell and pretend
that they do nothing. For instance, why do we have:

\begin{verbatim}
putStr :: String -> IO ()
\end{verbatim}

rather than the simpler:

\begin{verbatim}
putStr :: String -> ()
\end{verbatim}

Two reasons: Haskell is lazy, so it would never call a function whose
output --- here, the unit object --- is not used for anything. And, even
if it weren't lazy, it could still freely change the order of such calls
and thus garble the output. The only way to force sequential execution
of two functions in Haskell is through data dependency. The input of one
function must depend on the output of another. Having \code{RealWorld}
passed between \code{IO} actions enforces sequencing.

Conceptually, in this program:

\begin{verbatim}
main :: IO () main = do putStr "Hello " putStr "World!"
\end{verbatim}

the action that prints ``World!'' receives, as input, the Universe in
which ``Hello '' is already on the screen. It outputs a new Universe,
with ``Hello World!'' on the screen.

\section{Conclusion}\label{conclusion}

Of course I have just scratched the surface of monadic programming.
Monads not only accomplish, with pure functions, what normally is done
with side effects in imperative programming, but they also do it with a
high degree of control and type safety. They are not without drawbacks,
though. The major complaint about monads is that they don't easily
compose with each other. Granted, you can combine most of the basic
monads using the monad transformer library. It's relatively easy to
create a monad stack that combines, say, state with exceptions, but
there is no formula for stacking arbitrary monads together.

Next:
\href{https://bartoszmilewski.com/2016/12/27/monads-categorically/}{Monads
Categorically}.
